\paragraph*{Ongoing work}
\label{sec:conclusion}

%% Now let's review our tasks listed in Section 3. For task 1, we applied a NLP model and different kernel methods to summary the simulations as much as we can to get a overview quantitatively. For task 2, we build a context visulazation to illustrate the details of the specific simulation. The design principle of this view is to remain the ground truth as much as possible. For task3, we follow the focus and context design, which allows user to spot the earthquake simulation they are interested and explore it in the context view. The prototype looks promising for now, since it provides a way to explore the whole earthquake dataset. In the meanwhile, there are  a lot of points for us to complete.

Even with the matrix diagram visualizations, the screen becomes cluttered as the number of simulation gets larger. In the future, when we get thousands of simulations, the matrix diagram will not scale. It is also worth considering if it's possible to cluster earthquakes hierarchically to maintain visual scalability. The way we choose periods for segmentation is also clearly inappropriate; we are currently investigating how to enable our machine learning methods with multiple overlapping motifs.
Finally, even though we have worked with domain experts in developing these tools, a thorough validation of the designs remains to be done.
