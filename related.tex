\section{Related Work}
\label{sec:related}

Although the problem of understanding how a building behaves during an earthquake has long been studied.
In order to investigate the performance of a building on a shake table, in 2007, engineers built a seven-story building for testing and re-creation of a seismic response of it based on recorded data of a full scale shake table test~\cite{Chourasia:2007:DRS:1247238.1247243}. Here, we are concerned about the relationship between the different physical measurements, as well as with the understanding of how these patterns are similar or different across different earthquakes.

At the same time, visual exploration of multivariate data sets is an integral part of scientific visualization. As in most real world phenomena, there exist multiple factors associated with the complex interactions of different variables. To gain an in-depth understanding of a scientific process, the relationship among the variables needs to be thoroughly investigated. Biswas et al.~\cite{Biswas:2013:AIFEMDS:1077-2626} propose a framework to classify isocontours of variables based on the relationship between them and users can explore the multivariate data sets using their interface. Comparing time series data is another great challenge. Kernel methods, such as Support Vector Machines and Gaussian Process have become classical data-analysis tools. Vert et al. propose an alignment kernel for time series which is widely used when comparing time series~\cite{DBLP:journals/corr/abs-cs-0610033}.


